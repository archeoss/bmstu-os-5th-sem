% !TeX document-id = {c68f4be8-c497-43e0-82df-e9ebfbea9577}
% !TeX TXS-program:pdflatex = pdflatex -synctex=1 -interaction=nonstopmode --shell-escape %.tex
% новая команда \RNumb для вывода римских цифр
\documentclass[a4paper,12pt]{bmstu}
\usepackage{booktabs}

\usepackage{cmap} % Улучшенный поиск русских слов в полученном pdf-файле
\usepackage[T2A]{fontenc} % Поддержка русских букв
\usepackage[utf8]{inputenc} % Кодировка utf8
\usepackage[english,russian]{babel} % Языки: русский, английский
%\usepackage{pscyr} % Нормальные шрифты
\usepackage{amsmath}
\usepackage{geometry}
\geometry{left=30mm}
\geometry{right=15mm}
\geometry{top=20mm}
\geometry{bottom=20mm}
\usepackage{titlesec}
\titleformat{\section}
{\normalsize\bfseries}
{\thesection}
{1em}{}
\titlespacing*{\chapter}{0pt}{-30pt}{8pt}
\titlespacing*{\section}{\parindent}{*4}{*4}
\titlespacing*{\subsection}{\parindent}{*4}{*4}
\usepackage{setspace}
\onehalfspacing % Полуторный интервал
\frenchspacing
\usepackage{indentfirst} % Красная строка
\usepackage{titlesec}
\titleformat{\chapter}{\LARGE\bfseries}{\thechapter}{20pt}{\LARGE\bfseries}
\titleformat{\section}{\Large\bfseries}{\thesection}{20pt}{\Large\bfseries}
\usepackage{listings}
\lstdefinestyle{customasm}{
    belowcaptionskip=1\baselineskip,
    frame=single, 
    frameround=tttt,
    xleftmargin=\parindent,
    language=[x86masm]Assembler,
    basicstyle=\footnotesize\ttfamily,
    commentstyle=\itshape\color{green!60!black},
    keywordstyle=\color{blue!80!black},
    identifierstyle=\color{red!80!black},
    tabsize=4,
    numbers=left,
    numbersep=8pt,
    stepnumber=1,
    numberstyle=\tiny\color{gray}, 
    columns = fullflexible,
}
\usepackage{xcolor}


\begin{document}

\makereporttitle
    {Информатика, искусственный интеллект и системы управления} % Название факультета
    {Программное обеспечение ЭВМ и информационные технологии} % Название кафедры
    {лабораторной работе №~1} % Название работы (в дат. падеже)
    {Операционные системы} % Название курса (необязательный аргумент)
    {Обработчик прерывания от системного таймера} % Тема работы
    {} % Номер варианта (необязательный аргумент)
    {ИУ7-55Б} % Номер группы
    {Романов~С.~К.} % ФИО студента
    {Рязанова~Н.~Ю.} % ФИО преподавателя
\clearpage

\tableofcontents

\chapter{Исходный дизассемблированный код}\label{ch:source}
\section{Прерывание 8h}\label{sec:Dis_int}
\begin{lstlisting}[style={asm},label={lst:INT8H}]
    Temp.lst						 Sourcer Listing v3.07     6-Sep-22   7:15 pm   Page 1

    ;; вызов sub_1
    020A:0746  E8 0070				call	sub_6			; (07B9)
    ;; Сохранение регистров ES, DS, AX, DX
    020A:0749  06					push	es
    020A:074A  1E					push	ds
    020A:074B  50					push	ax
    020A:074C  52					push	dx

    ;; DS = 0040
    020A:074D  B8 0040				mov	ax,40h
    020A:0750  8E D8				mov	ds,ax
    ;; AX = 0
    020A:0752  33 C0				xor	ax,ax
    020A:0754  8E C0				mov	es,ax

    ;; 0040:006Ch - адрес счетчикa таймера
    020A:0756  FF 06 006C				inc	word ptr ds:[6Ch]	; (0040:006C=41B9h)
    020A:075A  75 04				jnz	loc_2			; Jump if not zero

    ;; 0040:006Eh - старшие 2 байта счетчикa таймера
    020A:075C  FF 06 006E				inc	word ptr ds:[6Eh]	; (0040:006E=13h)
    020A:0760			loc_2:

    ;; Проверка: 0040:006Eh == 18h (24) и 0040:006Ch == B0h (176)
    ;; Можно убедиться в том, что: 18h << 16 + B0h = 24 * 60 * 60 * freq,
    ;; где freq - кол-во раз, которое вызывается таймер в секунду.
    ;; Таким образом из того, что условие выполняется, следует, что прошли сутки.

    020A:0760  83 3E 006E 18			cmp	word ptr ds:[6Eh],18h	; (0040:006E=13h)
    020A:0765  75 15				jne	loc_3			; Jump if not equal
    020A:0767  81 3E 006C 00B0			cmp	word ptr ds:[6Ch],0B0h	; (0040:006C=41B9h)
    020A:076D  75 0D				jne	loc_3			; Jump if not equal

    ;; Обнуление счетчика (старшего слова и младшего слова)

    020A:076F  A3 006E				mov	word ptr ds:[6Eh],ax	; (0040:006E=13h)
    020A:0772  A3 006C				mov	word ptr ds:[6Ch],ax	; (0040:006C=41B9h)

    ;; Прошло более 24 часов, занесение значения 1 в 0040:0070

    020A:0775  C6 06 0070 01			mov	byte ptr ds:[70h],1	; (0040:0070=0)

    ;; AL = 8
    020A:077A  0C 08				or	al,8

    020A:077C			loc_3:
    ;; cохранение AX
    020A:077C  50					push	ax

    ;; Декремент счетчика отключения моторчика
    020A:077D  FE 0E 0040				dec	byte ptr ds:[40h]	; (0040:0040=6Ah)
    020A:0781  75 0B				jnz	loc_4			; Jump if not zero

    ;; Установка флага отключения моторчика дисковода (1-3 биты == 0)
    020A:0783  80 26 003F F0			and	byte ptr ds:[3Fh],0F0h	; (0040:003F=0)

    ;; 3 строчки - посылка команды отключения дисководу
    020A:0788  B0 0C				mov	al,0Ch
    020A:078A  BA 03F2				mov	dx,3F2h
    020A:078D  EE					out	dx,al			; port 3F2h, dsk0 contrl output

    020A:078E			loc_4:
    ;; Восстановление AX
    020A:078E  58					pop	ax

    ;; Проверка третьего бита (Parity Flag)
    020A:078F  F7 06 0314 0004			test	word ptr ds:[314h],4	; (0040:0314=3200h)
    020A:0795  75 0C				jnz	loc_5			; Jump if not zero

    ;; Копирование младшего байта FLAGS в ah
    020A:0797  9F					lahf				; Load ah from flags
    ;; Смена мест:
    ;; теперь в ax: 08XXh - где XX - младший байт FLAGS
    020A:0798  86 E0				xchg	ah,al
    ;; Кладем это на стек и вызываем прерывание
    020A:079A  50					push	ax

    ;; Вызываем 1Сh через адрес в таблице векторов. До этого мы добавили в стек AX, в то время как
    ;; вызов int делает push флагов (то есть наш ax, описанный 6 строками выше будет как FLAGS в 1Ch)
    020A:079B  26: FF 1E 0070			call	dword ptr es:[70h]	; (0000:0070=6ADh)
    020A:07A0  EB 03				jmp	short loc_6		; (07A5)
    020A:07A2  90					nop
    020A:07A3			loc_5:
    020A:07A3  CD 1C				int	1Ch			; Timer break (call each 18.2ms)
    020A:07A5			loc_6:
    020A:07A5  E8 0011				call	sub_6			; (07B9)

    ;; Сброс контроллера прерываний
    020A:07A8  B0 20				mov	al,20h			; ' '
    020A:07AA  E6 20				out	20h,al			; port 20h, 8259-1 int command
                                           ;  al = 20h, end of interrupt
										;  al = 20h, end of interrupt

    ;; Восстановление регистров
    020A:07AC  5A					pop	dx
    020A:07AD  58					pop	ax
    020A:07AE  1F					pop	ds
    020A:07AF  07					pop	es
    020A:07B0  E9 FE99				jmp	$-164h
    ...
    ;; 07B0h - 0164h = 064Ch
    ;; Листинг ниже
    ...

\end{lstlisting}
\clearpage

\section{SUB\_6}\label{sec:sub_6}
\begin{lstlisting}[style={asm},label={lst:SUB_6}]
                        sub_6		proc	near
    ;; Сохранение регистров
    020A:07B9  1E					push	ds
    020A:07BA  50					push	ax
    ;; DS = 40h
    020A:07BB  B8 0040				mov	ax,40h
    020A:07BE  8E D8				mov	ds,ax

    ;; Младший байт FLAGS в AH
    020A:07C0  9F					lahf				; Load ah from flags

    ;; Установлены ли старший бит IOPL или DF?
    020A:07C1  F7 06 0314 2400			test	word ptr ds:[314h],2400h	; (0040:0314=3200h)
    020A:07C7  75 0C				jnz	loc_8			; Jump if not zero


    ;; сброс IF (Iterrupt flag) в 0040:0314h (обнуление 9 бита)
    020A:07C9  F0> 81 26 0314 FDFF	           lock	and	word ptr ds:[314h],0FDFFh	; (0040:0314=3200h)

    020A:07D0			loc_7:
    ;; AH копируется в младший байт FLAGS
    020A:07D0  9E					sahf				; Store ah into flags
    020A:07D1  58					pop	ax
    020A:07D2  1F					pop	ds
    020A:07D3  EB 03				jmp	short loc_9		; (07D8)

    020A:07D5			loc_8:
    ;; Сброс IF (Iterrupt flag)
    020A:07D5  FA					cli				; Disable interrupts
    020A:07D6  EB F8				jmp	short loc_7		; (07D0)
    020A:07D8			loc_9:
    020A:07D8  C3					retn
                        sub_6		endp
\end{lstlisting}

\clearpage

\chapter{Схема алгоритмов}\label{ch:alg}
\section{Схема прерывания int 8h}\label{sec:int}
\includeimage
{diagram_A.png} % Имя файла без расширения (файл должен быть расположен в директории inc/img/)
{f} % Обтекание (без обтекания)
{h} % Положение рисунка (см. figure из пакета float)
{0.5\textwidth} % Ширина рисунка
{Схема А} % Подпись рисунка
\clearpage

\includeimage
{diagram_B.png} % Имя файла без расширения (файл должен быть расположен в директории inc/img/)
{f} % Обтекание (без обтекания)
{h} % Положение рисунка (см. figure из пакета float)
{0.75\textwidth} % Ширина рисунка
{Схема Б} % Подпись рисунка
\clearpage

\includeimage
{diagram_C.png} % Имя файла без расширения (файл должен быть расположен в директории inc/img/)
{f} % Обтекание (без обтекания)
{h} % Положение рисунка (см. figure из пакета float)
{0.65\textwidth} % Ширина рисунка
{Схема С} % Подпись рисунка
\clearpage

\section{Схема подпрограммы sub\_6}\label{sec:sub}
\includeimage
{diagram_sub} % Имя файла без расширения (файл должен быть расположен в директории inc/img/)
{f} % Обтекание (без обтекания)
{h} % Положение рисунка (см. figure из пакета float)
{0.65\textwidth} % Ширина рисунка
{sub\_6} % Подпись рисунка

\end{document}
