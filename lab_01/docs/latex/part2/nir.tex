% !TEX program = xelatex
\usepackage{cmap} % Улучшенный поиск русских слов в полученном pdf-файле
\usepackage[T2A]{fontenc} % Поддержка русских букв
\usepackage[utf8]{inputenc} % Кодировка utf8
\usepackage[english,russian]{babel} % Языки: русский, английский
%\usepackage{pscyr} % Нормальные шрифты
\usepackage{amsmath}
\usepackage{geometry}
\geometry{left=30mm}
\geometry{right=15mm}
\geometry{top=20mm}
\geometry{bottom=20mm}
\usepackage{titlesec}
\titleformat{\section}
{\normalsize\bfseries}
{\thesection}
{1em}{}
\titlespacing*{\chapter}{0pt}{-30pt}{8pt}
\titlespacing*{\section}{\parindent}{*4}{*4}
\titlespacing*{\subsection}{\parindent}{*4}{*4}
\usepackage{setspace}
\onehalfspacing % Полуторный интервал
\frenchspacing
\usepackage{indentfirst} % Красная строка
\usepackage{titlesec}
\titleformat{\chapter}{\LARGE\bfseries}{\thechapter}{20pt}{\LARGE\bfseries}
\titleformat{\section}{\Large\bfseries}{\thesection}{20pt}{\Large\bfseries}
\usepackage{listings}
\lstdefinestyle{customasm}{
    belowcaptionskip=1\baselineskip,
    frame=single, 
    frameround=tttt,
    xleftmargin=\parindent,
    language=[x86masm]Assembler,
    basicstyle=\footnotesize\ttfamily,
    commentstyle=\itshape\color{green!60!black},
    keywordstyle=\color{blue!80!black},
    identifierstyle=\color{red!80!black},
    tabsize=4,
    numbers=left,
    numbersep=8pt,
    stepnumber=1,
    numberstyle=\tiny\color{gray}, 
    columns = fullflexible,
}
\usepackage{xcolor}


\begin{document}
	
\include{title.tex}
\pagenumbering{gobble}

\normalsize

\pagenumbering{arabic}
\setcounter{page}{2}

\clearpage

\section{Функции обработчика прерывания от системного таймера}

Обработчик прерываний от системного таймера имеет наивысший приоритет. Никакая другая работа в системе не может выполняться во время обработчика прерывания от системного таймера. Обработчик прерывания от системного таймера должен завершаться как можно быстрее, чтобы не влиять на отзывчивость системы (то, насколько быстро система отвечает на запросы пользователей).

\textbf{Тик} --- период времени между двумя последующими прерываниями таймера.

\textbf{Главный тик} --- период времени, равный N тикам таймера (число N зависит от конкретного варианта системы).

\textbf{Квант времени} --- временной интервал, в течение которого процесс мо-
жет использовать процессор до вытеснения другим процессом.

\subsection{Windows}

\textbf{По тику}:
\begin{itemize}[label=---]
	\item инкремент счётчика реально времени;
	\item декремент остатка кванта текущего потока;
	\item декремент счётчиков отложенных задач;
	\item если активен механизм профилирования ядра, то инициализация отложенного вызова обработчика ловушки профилирования ядра добавлением объекта в очередь DPC ( отложенных вызовов процедур). Обработчик ловушки профилирования регистрирует адрес команды, выполнявшейся на момент прерывания. 
\end{itemize}

\textbf{По главному тику:} 
\begin{itemize}[label=---]
	\item освобождение объекта «событие», которое ожидает диспетчер настройки баланса (один раз в секунду диспетчер настройки баланса сканирует
	очередь готовых потоков и повышает приоритет тех из них, которые находятся в состоянии ожидания около 4 секунд).
\end{itemize}

\textbf{По кванту:}
\begin{itemize}[label=---]
	\item инициация диспетчеризации потоков добавлением соответствующего объекта в очередь DPC.
\end{itemize}

\subsection{Unix}

% 187 вахалия
По тику:

\begin{itemize}[label=---]
	\item инкремент счётчика реального времени;
	\item декремент кванта текущего потока.
	\item обновление статистики использования процессора текущим процессом (инкрементмент поля p\_cpu дескриптора текущего процесса до максимального значения, равного 127);
	\item декремент счетчика времени, оставшегося до отправления на выполнение отложенных вызовов, при достижении нулевого значения счетчика -- выставление флага, указывающего на необходимость запуска обработчика отложенного вызова.
\end{itemize}

\textbf{По главному тику:} 
\begin{itemize}[label=---]
	\item регистрация отложенных вызовов функций, относящихся к работе планировщика,
	таких как пересчет приоритетов;
	\item пробуждение (то есть регистрирация отложенного вызова процедуры wakeup, которая перемещает дескриптор процесса из списка “спящих” в очередь "готовых к выполнению") системных процессов {\ttfamily swapper} и {\ttfamily pagedaemon};
	\item декремент счётчика времени, оставшегося до посылки одного из следующих сигналов:\clearpage
	\begin{enumerate}
		\item {\ttfamily SIGALRM} – сигнал, посылаемый процессу по истечении времени, заданного функцией {\ttfamily alarm()} (будильник реального времени);
		\item {\ttfamily SIGPROF} –  сигнал, посылаемый процессу по истечении времени, заданного в таймере профилирования (будильник профиля процесса);
		\item {\ttfamily SIGVTALRM} –  сигнал, посылаемый процессу по истечении времени работы в режиме задачи (будильник виртуального времени).
	\end{enumerate}
\end{itemize}

\textbf{По кванту:}
\begin{itemize}[label=---]
	\item если текущий процесс превысил выделенный ему квант процессорного времени, отправка ему сигнала SIGXCPU. По получению сигнала обработчик сигнала прерывает выполнение процесса.
\end{itemize}
\pagebreak

\section{Пересчёт динамических приоритетов}

Динамические приоритеты могут быть только у пользовательских процессов. В операционных системах процесс является владельцем ресурсов, в том числе -- владельцем приоритета.

\subsection{Windows}
В Windows приоритет назначается процессу. Исходный базовый приоритет потока наследуется от базового приоритета процесса. Реализуется приоритетная, вытесняющая система планирования, при  которой всегда выполняется хотя бы один работоспособный (готовый) поток с самым высоким приоритетом.

После того как поток был выбран для запуска, он запускается на время, называемое квантом. Но поток может и не израсходовать свой квант времени: если становится готов к запуску другой поток с более высоким приоритетом, текущий выполняемый поток может быть вытеснен.

Единого модуля под названием «планировщик» не существует. Процедуры, выполняющие обязанности по диспетчеризации, обобщенно называются диспетчером ядра.   Диспетчеризации потоков могут потребовать следующие события:

\begin{itemize}[label=---]
	\item поток становится готовым к выполнению;
	\item поток выходит из состояния выполнения из-за окончания его кванта времени;
	\item поток завершается или переходит в состояние ожидания;
	\item изменяется приоритет потока;
	\item изменяется родственность процессора потока.
\end{itemize}


Windows использует 32 уровня приоритета, от 0 до 31:
\begin{itemize}[label=---]
	\item шестнадцать уровней реального времени (от 16 до 31);
	\item шестнадцать изменяющихся уровней (от 0 до 15), из которых уровень 0 зарезервирован для потока обнуления страниц. 
\end{itemize}


Уровни приоритета потоков назначаются из двух разных позиций: от Windows API и от ядра Windows.

Windows API систематизирует процессы по классу приоритета, который присваивается им при их создании:
\begin{itemize}[label=---]
	\item реального времени (real-time, 4);
	\item высокий (high, 3);
	\item выше обычного (above normal, 7);
	\item обычный (normal, 2);
	\item ниже обычного (below normal, 5);
	\item простоя (idle, 1).
\end{itemize}

Затем назначается относительный приоритет потоков в рамках процессов. Здесь номера представляют изменение приоритета, применяющееся к базовому приоритету процесса:
\begin{itemize}[label=---]
	\item критичный по времени (time-critical, 15);
	\item наивысший (highest, 12);
	\item выше обычного (above-normal, 1);
	\item обычный (normal, 0);
	\item ниже обычного (below-normal,-1);
	\item самый низший (lowest, -2);
	\item простоя (idle, -15).
\end{itemize}

Уровень, критичный по времени, и уровень простоя (+15 и –15) называются уровнями насыщения и представляют конкретные применяемые уровни вместо смещений.

Таким образом, в Windows API каждый поток имеет базовый приоритет,
являющийся функцией класса приоритета процесса и его относительного приоритета процесса. Соответствие между приоритетами Windows API и ядра Windows приведено в таблице \ref{tbl:priority}.
\pagebreak
\begin{table}[h]
	\caption{Соответствие между приоритетами Windows API и ядра Windows}
	\begin{center}
		\begin{tabular}{|l|p{45pt}|p{45pt}|p{45pt}|p{45pt}|p{45pt}|p{45pt}|} 
			\hline
			{} & \textbf{real-time} & \textbf{high} & \textbf{above normal} & \textbf{normal} & \textbf{below normal} & \textbf{idle}\\
			\hline
			\textbf{time critical} & 31 & 15 & 15 & 15 & 15 & 15 \\
			\hline
			\textbf{highest} & 26 & 15 & 12 & 10 & 8 & 6 \\
			\hline
			\textbf{above normal} & 25 & 14 & 11 & 9 & 7 & 5 \\
			\hline
			\textbf{normal} & 24 & 13 & 10 & 8 & 6 & 4 \\
			\hline
			\textbf{below normal} & 23 & 12 & 9 & 7 & 5 & 3 \\
			\hline
			\textbf{lowest} & 22 & 11 & 8 & 6 & 4 & 2 \\
			\hline
			\textbf{idle} & 16 & 1 & 1 & 1 & 1 & 1 \\
			\hline
		\end{tabular}
	\end{center}
	\label{tbl:priority}
\end{table}

Исходный базовый приоритет потока наследуется от базового приоритета процесса. Процесс по умолчанию наследует свой базовый приоритет у того процесса, который его создал.

Сценарии повышения приоритета:
\begin{enumerate}
	\item Повышение вследствие событий планировщика или диспетчера (сокращение задержек).
	\item Повышение вследствие завершения ввода-вывода (сокращение задержек -- поток может вновь запуститься и начать новую операцию ввода-вывода). В таблице \ref{tab:io} приведены рекомендуемые значения повышения приоритета для устройств ввода-вывода.
	\clearpage
	\begin{table}[h!]
		\caption{Рекомендуемые значения повышения приоритета.}
		\begin{center}
			\begin{tabular}{|p{100mm}|l|}
				\hline
				\textbf{Устройство} & \textbf{Приращение} \\
				\hline
				Жесткий диск, привод компакт-дисков, параллельный порт, видеоустройство & 1 \\
				\hline
				Сеть, почтовый слот, именованный канал, последовательный порт & 2 \\
				\hline
				Клавиатура, мышь & 6 \\
				\hline
				Звуковое устройство & 8 \\
				\hline
			\end{tabular}
		\end{center}
		\label{tab:io}
	\end{table}
	\item Повышение вследствие ввода из пользовательского интерфейса (сокращение задержек и времени отклика).
	\item Повышение приоритета владельца блокировки.
	\item Повышение вследствие слишком продолжительного ожидания ресурса исполняющей системы (предотвращение зависания).
	\item Повышение в случае, когда готовый к запуску поток не был запущен в течение определенного времени (чтобы исключить бесконечное откладывание процессов).
	%табличку
	%\item Если поток ждал на семафоре, мьютексе или другом событии, то повышение при его освобождении.
	\item Повышение вследствие ожидания объекта ядра.
	\item Повышение приоритета потоков первого плана после ожидания (улучшение отзывчивости интерактивных приложений).
	\item Повышение приоритета после пробуждения GUI-потока (потоки-владельцы окон получают при пробуждении дополнительное повышение приоритета на 2).
	\item Повышение приоритетов для мультимедийных приложений и игр.
\end{enumerate}

Рассмотрим последние 2 сценария подробнее. 

\subsubsection*{Диспетчер настройки баланса}

Один раз в секунду сканируется очередь готовых потоков в поиске тех из них, которые находятся в состоянии ожидания около 4 секунд. Если такой поток будет найден, диспетчер настройки баланса повышает его приоритет до 15 единиц и устанавливает квантовую цель эквивалентной тактовой частоте процессора при подсчете 3 квантовых единиц. Как только квант истекает, приоритет потока тут же снижается до обычного базового приоритета. Если поток не был завершен и есть готовый к запуску поток с более высоким уровнем приоритета, поток с пониженным приоритетом возвращается в очередь готовых потоков.

Для минимизации времени своей работы, диспетчер настройки баланса сканирует только 16 готовых потоков. Если на данном уровне приоритета имеется больше потоков, он запоминает то место, на котором остановился, и начинает с него при следующем проходе очереди. Кроме того, он за один проход повысит приоритет только 10 потоков. Если найдет 10 потоков, заслуживающих именно этого повышения, он прекратит сканирование на этом месте и начнет его с этого же места при следующем проходе.

\subsubsection*{MMCSS}

В клиентские версии Windows была внедрена служба MMCSS, чьей целью было гарантировать проигрывание мультимедийного контента приложений, зарегистрированных с этой службой, без каких-либо сбоев.

MMCSS работает с определенными задачами, включая следующие: аудио, захват, распределение, игры, проигрывание, аудио профессионального качества, задачи администратора многооконного режима.

Каждая из этих задач включает информацию о свойствах, отличающих
их друг от друга. Одно из наиболее важных свойств для планирования потоков называется категорией планирования, что является первичным фактором, определяющим приоритет потоков, зарегистрированных с MMCSS. В таблице 3 показаны различные категории планирования.

\begin{table}[h]
	\caption{Категории планирования}
	\begin{center}
		\begin{tabular}{| p{90pt} | c | p{245pt} |} 
			\hline
			\textbf{Категория} & \textbf{Приоритет} & \textbf{Описание}\\
			\hline
			{High (Высокая) } & 23-26  & {Потоки профессионального аудио
				(Pro Audio), запущенные с приоритетом выше, чем у других потоков
				на системе, за исключением критических системных потоков} \\
			\hline
			{Medium (Средняя)} & 16-22  & {Потоки, являющиеся частью приложений первого плана, например
				Windows Media Player}\\
			\hline
			{Low (Низкая)} & 8-15 & {Все остальные потоки, не являющиеся частью предыдущих категорий} \\
			\hline
			{Exhausted (Исчерпавших потоков)} & 1-7 & {Потоки, исчерпавшие свою долю
				времени центрального процессора,
				выполнение которых продолжиться,
				только если не будут готовы к выполнению другие потоки с более высоким уровнем приоритета}\\
			\hline
		\end{tabular}
	\end{center}
\end{table}


Механизм, положенный в основу MMCSS, повышает приоритет потоков
внутри зарегистрированного процесса до уровня, соответствующего их категории планирования и относительного приоритета внутри этой категории на гарантированный срок. Затем он снижает категорию этих потоков до Exhausted, чтобы другие, не относящиеся к мультимедийным приложениям потоки, также могли получить ресурс.

\subsection{Unix}

В современных системах Unix ядро является вытесняющим – процесс в режиме ядра может быть вытеснен более приоритетным процессом, также находящимся в режиме ядра. Это сделано для того, чтобы система могла обслуживать процессы реального времени, такие как  аудио и видео.

Согласно приоритетам процессов и принципу вытесняющего циклического планирования формируется очередь готовых к выполнению процессов. В первую очередь выполняются процессы с большим приоритетом. Процессы с одинаковыми приоритетами выполняются в течении кванта времени, циклически друг за другом.


Приоритет задается любым целым числом, лежащим в диапазоне от 0 до 127 (чем меньше число, тем выше приоритет). Приоритеты от 0 до 49 зарезервированы для ядра, они являются фиксированными величинами. Прикладные процессы могут обладать приоритетом в диапазоне 50-127.


В традиционных системах Unix приоритет процесса определяется двумя факторами: 

\begin{itemize}[label=---]
	\item фактор <<любезности>> – целое число в диапазоне от 0 до 39 со значением 20 по умолчанию. Чем меньше значение фактора любезности, тем выше приоритет процесса. Пользователи могут повлиять на приоритет процесса при помощи изменения этого фактора, используя системный вызов {\ttfamily nice} (но только суперпользователь имеет полномочия увеличивать приоритет процесса);
	
	\item фактор утилизации - степень загруженности CPU в момент последнего обслуживания им процесса. Этот фактор позволяет системе динамически изменять приоритет процесса.
\end{itemize}

Структура \code{proc} содержит следующие поля, относящиеся к приоритетам:
\begin{itemize}[label=---]
	\item \code{p{\_}pri} -- текущий приоритет планирования;
	\item \code{p{\_}usrpri} -- приоритет режима задачи;
	\item \code{p{\_}cpu} -- результат последнего измерения использования процессора;
	\item \code{p{\_}nice} -- фактор <<любезности>>, устанавливаемый пользователем.
\end{itemize}

Планировщик использует p\_pri для принятия решения о том, какой процесс направить на выполнение. У процесса, находящегося в режиме задачи, значения p\_pri и p\_usrpri идентичны. Значение текущего приоритета p\_pri может быть повышено планировщиком для выполнения процесса в режиме ядра. Когда процесс просыпается после блокировки в системном вызове, его приоритет временно повышается. При создании процесса поле p\_cpu инициализируется нулем. На каждом тике обработчик таймера увеличивает поле p\_cpu текущего процесса на единицу, до максимального значения, равного 127.

Планировщик использует \code{p{\_}usrpri} для хранения приоритета, который будет назначен процессу при переходе из режима ядра в режим задачи. Ядро связывает приоритет сна (0--49) с событием или ожидаемым ресурсом, из-за которого процесс может быть заблокирован. Когда блокированный процесс просыпается, ядро устанавливает \code{p{\_}pri}, равное приоритету сна события или ресурса, на котором он был заблокирован.

В таблицах \ref{tab:bsd} (источник - книга «Unix изнутри», Ю. Вахалия.) и \ref{tbl:sleeppriority} (источник - книга «Операционная система UNIX», А. Робачевский) приведены значения приоритетов сна для систем \textbf{4.3BSD UNIX} и \textbf{SCO UNIX}.

\newpage
\begin{table}[h]
	\caption{Приоритеты сна в системе 4.3BSD UNIX}
	\label{tab:bsd}
	\begin{center}
		\begin{tabular}{ |c|c|c|  }
			\hline
			\textbf{Приоритет} & \textbf{Значение} & \textbf{Описание} \\
			\hline
			\texttt{PSWP} & 0 & Свопинг \\
			\hline
			\texttt{PSWP + 1} & 1 & Страничный демон \\
			\hline
			\texttt{PSWP + 1/2/4} & 1/2/4 & Другие действия по обработке памяти \\
			\hline
			\texttt{PINOD} & 10 & Ожидание освобождения inode \\
			\hline
			\texttt{PRIBIO} & 20 & Ожидание дискового ввода-вывода \\
			\hline
			\texttt{PRIBIO + 1} & 21 & Ожидание освобождения буфера \\
			\hline
			\texttt{PZERO} & 25 & Базовый приоритет \\
			\hline
			\texttt{TTIPRI} & 28 & Ожидание ввода с терминала \\
			\hline
			\texttt{TTOPRI} & 29 & Ожидание вывода с терминала \\
			\hline 
			\texttt{PWAIT} & 30 & Ожидание завершения процесса потомка \\
			\hline
			\texttt{PLOCK} & 35 & Консультативное ожидание блокир. ресурса \\
			\hline
			\texttt{PSLEP} & 40 & Ожидание сигнала \\
			\hline
		\end{tabular}
	\end{center}
\end{table}

\begin{table}[h]
	\caption{Приоритеты сна в системах 4.3BSD UNIX и SCO UNIX}
	\begin{center}
		\begin{tabular}{|l|p{75pt}|p{75pt}|} 
			\hline
			\textbf{Событие} & \textbf{Приоритет 4.3BSD UNIX} & \textbf{Приоритет SCO UNIX}\\
			\hline
			{Ожидание загрузки страницы/сегмента} & 0 & 95\\
			\hline
			{Ожидание индексного дескриптора} & 10 & 88\\
			\hline
			{Ожидание ввода/вывода} & 20 & 81 \\
			\hline
			{Ожидание буфера} & 30 & 80\\
			\hline
			{Ожидание терминального ввода} &    & 75\\
			\hline
			{Ожидание терминального вывода} &    & 74\\
			\hline
			{Ожидание завершения выполнения} &    & 73\\
			\hline
			{Ожидание события} & 40 & 66\\
			\hline
		\end{tabular}
	\end{center}
	\label{tbl:sleeppriority}
\end{table}


Каждую секунду обработчик прерывания таймера инициализирует отложенный вызов процедуры {\ttfamily schedcpu()}, которая уменьшает значение {\ttfamily p{\_}cpu} каждого процесса исходя из фактора <<полураспада>> (decay factor). В {\ttfamily 4.3BSD} для расчета полураспада применяется следующая формула:
\[
d = \frac{2\cdot load{\_}average}{2\cdot load{\_}average + 1},
\]
где  {\ttfamily load{\_}average} -- это среднее количество процессов в состоянии готовности к выполнению, за последнюю секунду. Процедура {\ttfamily schedcpu()} также пересчитывает приоритеты режима задачи всех процессов по формуле
\[
{p\_usrpri} = PUSER + \frac{p{\_}cpu}{4} + 2\cdot {p{\_}nice},
\]
где {\ttfamily PUSER} -- базовый приоритет в режиме задачи, равный 50.

Таким образом, если процесс до вытеснения другим процессом использовал большое количество процессорного времени, его {\ttfamily p{\_}cpu} будет увеличен, что приведет к увеличению значения {\ttfamily p{\_}usrpri} и, следовательно, к понижению приоритета. Чем дольше процесс простаивает в очереди на выполнение, тем меньше его {\ttfamily p{\_}cpu}, и, соответственно, выше приоритет. 

Такая схема позволяет исключить бесконечное откладывание низкоприоритетных процессов. При ее применении процессы, которые осуществляют много операций ввода-вывода, получают преимущество, так как они проводят много времени в ожидании завершения этих операций. Напротив, те процессы, что производят много вычислений, используют большое количество процессорного времени, и их приоритет после вытеснения будет понижен. 

\section*{ВЫВОД}

Семейство ОС Windows и семейство OC Unix оба являются системами разделения времени с динамическими приоритетами и вытеснением. В связи с общим подходом обработчик прерывания от системного таймера выполняет в этих системах схожие функции: 

\begin{itemize}[label=---]
	\item выполняет декремент кванта;
	\item инициализурует отложенные вызовы функций, относящихся к работе планировщика (например, пересчёт динамических приоритетов);
	\item декрементирует счетчики времени: часов, таймеров, будильников реального времени, а также счетчики времени отложенных действий.
\end{itemize}

Для того чтобы исключить бесконечное откладывание пользовательских процессов, повысить уровень отзывчивости системы и поддерживать процессы реального времени, такие как аудио и видео, в семействах ОС Windows и Unix выполняется пересчет их динамических приоритетов, однако реализация пересчета в этих семействах различается. 

\end{document}
