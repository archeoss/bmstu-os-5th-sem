% !TeX document-id = {c68f4be8-c497-43e0-82df-e9ebfbea9577}
% !TeX TXS-program:pdflatex = pdflatex -synctex=1 -interaction=nonstopmode --shell-escape %.tex
% новая команда \RNumb для вывода римских цифр
\documentclass[a4paper,12pt]{bmstu}
\usepackage{booktabs}


\usepackage{cmap} % Улучшенный поиск русских слов в полученном pdf-файле
\usepackage[T2A]{fontenc} % Поддержка русских букв
\usepackage[utf8]{inputenc} % Кодировка utf8
\usepackage[english,russian]{babel} % Языки: русский, английский
%\usepackage{pscyr} % Нормальные шрифты
\usepackage{amsmath}
\usepackage{geometry}
\geometry{left=30mm}
\geometry{right=15mm}
\geometry{top=20mm}
\geometry{bottom=20mm}
\usepackage{titlesec}
\titleformat{\section}
{\normalsize\bfseries}
{\thesection}
{1em}{}
\titlespacing*{\chapter}{0pt}{-30pt}{8pt}
\titlespacing*{\section}{\parindent}{*4}{*4}
\titlespacing*{\subsection}{\parindent}{*4}{*4}
\usepackage{setspace}
\onehalfspacing % Полуторный интервал
\frenchspacing
\usepackage{indentfirst} % Красная строка
\usepackage{titlesec}
\titleformat{\chapter}{\LARGE\bfseries}{\thechapter}{20pt}{\LARGE\bfseries}
\titleformat{\section}{\Large\bfseries}{\thesection}{20pt}{\Large\bfseries}
\usepackage{listings}
\usepackage{xcolor}

\lstdefinestyle{asm}{
    language={[x86masm]Assembler},
    backgroundcolor=\color{white},
    basicstyle=\footnotesize\ttfamily,
    keywordstyle=\color{blue},
    stringstyle=\color{red},
    commentstyle=\color{gray},
    numbers=left,
    numberstyle=\tiny,
    stepnumber=1,
    numbersep=5pt,
    frame=single,
    tabsize=4,
    captionpos=b,
    breaklines=true
}


\lstset{
    literate=
        {а}{{\selectfont\char224}}1
        {б}{{\selectfont\char225}}1
        {в}{{\selectfont\char226}}1
        {г}{{\selectfont\char227}}1
        {д}{{\selectfont\char228}}1
        {е}{{\selectfont\char229}}1
        {ё}{{\"e}}1
        {ж}{{\selectfont\char230}}1
        {з}{{\selectfont\char231}}1
        {и}{{\selectfont\char232}}1
        {й}{{\selectfont\char233}}1
        {к}{{\selectfont\char234}}1
        {л}{{\selectfont\char235}}1
        {м}{{\selectfont\char236}}1
        {н}{{\selectfont\char237}}1
        {о}{{\selectfont\char238}}1
        {п}{{\selectfont\char239}}1
        {р}{{\selectfont\char240}}1
        {с}{{\selectfont\char241}}1
        {т}{{\selectfont\char242}}1
        {у}{{\selectfont\char243}}1
        {ф}{{\selectfont\char244}}1
        {х}{{\selectfont\char245}}1
        {ц}{{\selectfont\char246}}1
        {ч}{{\selectfont\char247}}1
        {ш}{{\selectfont\char248}}1
        {щ}{{\selectfont\char249}}1
        {ъ}{{\selectfont\char250}}1
        {ы}{{\selectfont\char251}}1
        {ь}{{\selectfont\char252}}1
        {э}{{\selectfont\char253}}1
        {ю}{{\selectfont\char254}}1
        {я}{{\selectfont\char255}}1
        {А}{{\selectfont\char192}}1
        {Б}{{\selectfont\char193}}1
        {В}{{\selectfont\char194}}1
        {Г}{{\selectfont\char195}}1
        {Д}{{\selectfont\char196}}1
        {Е}{{\selectfont\char197}}1
        {Ё}{{\"E}}1
        {Ж}{{\selectfont\char198}}1
        {З}{{\selectfont\char199}}1
        {И}{{\selectfont\char200}}1
        {Й}{{\selectfont\char201}}1
        {К}{{\selectfont\char202}}1
        {Л}{{\selectfont\char203}}1
        {М}{{\selectfont\char204}}1
        {Н}{{\selectfont\char205}}1
        {О}{{\selectfont\char206}}1
        {П}{{\selectfont\char207}}1
        {Р}{{\selectfont\char208}}1
        {С}{{\selectfont\char209}}1
        {Т}{{\selectfont\char210}}1
        {У}{{\selectfont\char211}}1
        {Ф}{{\selectfont\char212}}1
        {Х}{{\selectfont\char213}}1
        {Ц}{{\selectfont\char214}}1
        {Ч}{{\selectfont\char215}}1
        {Ш}{{\selectfont\char216}}1
        {Щ}{{\selectfont\char217}}1
        {Ъ}{{\selectfont\char218}}1
        {Ы}{{\selectfont\char219}}1
        {Ь}{{\selectfont\char220}}1
        {Э}{{\selectfont\char221}}1
        {Ю}{{\selectfont\char222}}1
        {Я}{{\selectfont\char223}}1
}
%\lstset{basicstyle=\fontsize{10}{10}\selectfont,breaklines=true,inputencoding=utf8,extendedchars=\true}

\newcommand{\specchapter}[1]{\chapter*{#1}\addcontentsline{toc}{chapter}{#1}}
\newcommand{\specsection}[1]{\section*{#1}\addcontentsline{toc}{section}{#1}}
\newcommand{\specsubsection}[1]{\subsection*{#1}\addcontentsline{toc}{subsection}{#1}}
\newcommand{\RNumb}[1]{\uppercase\expandafter{\romannumeral #1\relax}}
\newcommand{\jj}{\righthyphenmin=20 \justifying}


\lstdefinestyle{asm}{
    language={[x86masm]Assembler},
    backgroundcolor=\color{white},
    basicstyle=\footnotesize\ttfamily,
    keywordstyle=\color{blue},
    stringstyle=\color{red},
    commentstyle=\color{gray},
    numbers=left,
    numberstyle=\tiny,
    stepnumber=1,
    numbersep=5pt,
    frame=single,
    tabsize=4,
    captionpos=b,
    breaklines=true
}

\lstset{basicstyle=\fontsize{10}{10}\selectfont,breaklines=true,inputencoding=utf8x,extendedchars=\true}

\newcommand{\specchapter}[1]{\chapter*{#1}\addcontentsline{toc}{chapter}{#1}}
\newcommand{\specsection}[1]{\section*{#1}\addcontentsline{toc}{section}{#1}}
\newcommand{\specsubsection}[1]{\subsection*{#1}\addcontentsline{toc}{subsection}{#1}}
\newcommand{\RNumb}[1]{\uppercase\expandafter{\romannumeral #1\relax}}
\newcommand{\jj}{\righthyphenmin=20 \justifying}

\begin{document}

\makereporttitle
    {Информатика, искусственный интеллект и системы управления} % Название факультета
    {Программное обеспечение ЭВМ и информационные технологии} % Название кафедры
    {лабораторной работе №~1} % Название работы (в дат. падеже)
    {Операционные системы} % Название курса (необязательный аргумент)
    {Обработчик прерывания от системного таймера} % Тема работы
    {} % Номер варианта (необязательный аргумент)
    {ИУ7-55Б} % Номер группы
    {Романов~С.~К.} % ФИО студента
    {Рязанова~Н.~Ю.} % ФИО преподавателя
\clearpage

%\stepcounter{chapter}
\chapter{Исходный дизассемблированный код}\label{ch:source}
\section{Прерывание 8h}\label{sec:Dis_int}
\begin{lstlisting}[style={asm},label={lst:INT8H}]
    Temp.lst						 Sourcer Listing v3.07     6-Sep-22   7:15 pm   Page 1

    020A:0746  E8 0070				call	sub_6			; (07B9)
    020A:0749  06					push	es
    020A:074A  1E					push	ds
    020A:074B  50					push	ax
    020A:074C  52					push	dx
    020A:074D  B8 0040				mov	ax,40h
    020A:0750  8E D8				mov	ds,ax
    020A:0752  33 C0				xor	ax,ax			; Zero register
    020A:0754  8E C0				mov	es,ax
    020A:0756  FF 06 006C				inc	word ptr ds:[6Ch]	; (0040:006C=41B9h)
    020A:075A  75 04				jnz	loc_2			; Jump if not zero
    020A:075C  FF 06 006E				inc	word ptr ds:[6Eh]	; (0040:006E=13h)
    020A:0760			loc_2:
    020A:0760  83 3E 006E 18			cmp	word ptr ds:[6Eh],18h	; (0040:006E=13h)
    020A:0765  75 15				jne	loc_3			; Jump if not equal
    020A:0767  81 3E 006C 00B0			cmp	word ptr ds:[6Ch],0B0h	; (0040:006C=41B9h)
    020A:076D  75 0D				jne	loc_3			; Jump if not equal
    020A:076F  A3 006E				mov	word ptr ds:[6Eh],ax	; (0040:006E=13h)
    020A:0772  A3 006C				mov	word ptr ds:[6Ch],ax	; (0040:006C=41B9h)
    020A:0775  C6 06 0070 01			mov	byte ptr ds:[70h],1	; (0040:0070=0)
    020A:077A  0C 08				or	al,8
    020A:077C			loc_3:
    020A:077C  50					push	ax
    020A:077D  FE 0E 0040				dec	byte ptr ds:[40h]	; (0040:0040=6Ah)
    020A:0781  75 0B				jnz	loc_4			; Jump if not zero
    020A:0783  80 26 003F F0			and	byte ptr ds:[3Fh],0F0h	; (0040:003F=0)
    020A:0788  B0 0C				mov	al,0Ch
    020A:078A  BA 03F2				mov	dx,3F2h
    020A:078D  EE					out	dx,al			; port 3F2h, dsk0 contrl output
    020A:078E			loc_4:
    020A:078E  58					pop	ax
    020A:078F  F7 06 0314 0004			test	word ptr ds:[314h],4	; (0040:0314=3200h)
    020A:0795  75 0C				jnz	loc_5			; Jump if not zero
    020A:0797  9F					lahf				; Load ah from flags
    020A:0798  86 E0				xchg	ah,al
    020A:079A  50					push	ax
    020A:079B  26: FF 1E 0070			call	dword ptr es:[70h]	; (0000:0070=6ADh)
    020A:07A0  EB 03				jmp	short loc_6		; (07A5)
    020A:07A2  90					nop
    020A:07A3			loc_5:
    020A:07A3  CD 1C				int	1Ch			; Timer break (call each 18.2ms)
    020A:07A5			loc_6:
    020A:07A5  E8 0011				call	sub_6			; (07B9)
    020A:07A8  B0 20				mov	al,20h			; ' '
    020A:07AA  E6 20				out	20h,al			; port 20h, 8259-1 int command
                                           ;  al = 20h, end of interrupt

\end{lstlisting}
\clearpage

\section{SUB\_6}\label{subsec:sub_6}
\begin{lstlisting}[style={asm},label={lst:SUB_6}]
                        sub_6		proc	near
    020A:07B9  1E					push	ds
    020A:07BA  50					push	ax
    020A:07BB  B8 0040				mov	ax,40h
    020A:07BE  8E D8				mov	ds,ax
    020A:07C0  9F					lahf				; Load ah from flags
    020A:07C1  F7 06 0314 2400			test	word ptr ds:[314h],2400h	; (0040:0314=3200h)
    020A:07C7  75 0C				jnz	loc_8			; Jump if not zero
    020A:07C9  F0> 81 26 0314 FDFF	           lock	and	word ptr ds:[314h],0FDFFh	; (0040:0314=3200h)
    020A:07D0			loc_7:
    020A:07D0  9E					sahf				; Store ah into flags
    020A:07D1  58					pop	ax
    020A:07D2  1F					pop	ds
    020A:07D3  EB 03				jmp	short loc_9		; (07D8)
    020A:07D5			loc_8:
    020A:07D5  FA					cli				; Disable interrupts
    020A:07D6  EB F8				jmp	short loc_7		; (07D0)
    020A:07D8			loc_9:
    020A:07D8  C3					retn
                        sub_6		endp
\end{lstlisting}

\clearpage

\section{Прерывание 1Ch}\label{sec:dis_1ch}
\begin{lstlisting}[style={asm},label={lst:1ch}]

    020A:06AD  EB 9D				jmp	short $-61h
    ...
    ; 06AD-61h == 064Ch
    ...
    020A:064C			loc_1:
    020A:064C  1E					push	ds
    020A:064D  50					push	ax
    020A:064E  B8 0040				mov	ax,40h
    020A:0651  8E D8				mov	ds,ax
    020A:0653  F7 06 0314 2400			test	word ptr ds:[314h],2400h	; (0040:0314=3200h)
    020A:0659  75 4F				jnz	loc_9			; Jump if not zero
    020A:065B  55					push	bp
    020A:065C  8B EC				mov	bp,sp
    020A:065E  8B 46 0A				mov	ax,[bp+0Ah]
    020A:0661  5D					pop	bp
    020A:0662  A9 0100				test	ax,100h
    020A:0665  75 43				jnz	loc_9			; Jump if not zero
    020A:0667  A9 0200				test	ax,200h
    020A:066A  74 22				jz	loc_5			; Jump if zero
    020A:066C  F0> 81 0E 0314 0200	           lock	or	word ptr ds:[314h],200h	; (0040:0314=3200h)
    020A:0673  F7 06 0314 0003			test	word ptr ds:[314h],3	; (0040:0314=3200h)
    020A:0679  75 2F				jnz	loc_9			; Jump if not zero
    020A:067B			loc_2:
    020A:067B  86 E0				xchg	ah,al
    020A:067D  FC					cld				; Clear direction
    020A:067E  A8 04				test	al,4
    020A:0680  75 25				jnz	loc_8			; Jump if not zero
    020A:0682			loc_3:
    020A:0682  A8 08				test	al,8
    020A:0684  75 11				jnz	loc_6			; Jump if not zero
    020A:0686  70 19				jo	loc_7			; Jump if overflow=1
    020A:0688			loc_4:
    020A:0688  9E					sahf				; Store ah into flags
    020A:0689  58					pop	ax
    020A:068A  1F					pop	ds
    020A:068B  CA 0002				retf	2			; Return far
    020A:068E			loc_5:
    020A:068E  F0> 81 26 0314 FDFF	           lock	and	word ptr ds:[314h],0FDFFh	; (020A:0314=3231h)
    020A:0695  EB E4				jmp	short loc_2		; (067B)
    020A:0697			loc_6:
    020A:0697  70 EF				jo	loc_4			; Jump if overflow=1
    020A:0699  50					push	ax
    020A:069A  B0 7F				mov	al,7Fh
    020A:069C  04 02				add	al,2
    020A:069E  58					pop	ax
    020A:069F  EB E7				jmp	short loc_4		; (0688)
    020A:06A1			loc_7:
    020A:06A1  50					push	ax
    020A:06A2  32 C0				xor	al,al			; Zero register
    020A:06A4  58					pop	ax
    020A:06A5  EB E1				jmp	short loc_4		; (0688)
    020A:06A7			loc_8:
    020A:06A7  FD					std				; Set direction flag
    020A:06A8  EB D8				jmp	short loc_3		; (0682)
    020A:06AA			loc_9:
    020A:06AA  58					pop	ax
    020A:06AB  1F					pop	ds
    020A:06AC  CF					iret				; Interrupt return
\end{lstlisting}
\clearpage
\chapter{Схема алгоритмов}\label{ch:alg}
\section{Схема прерывания int 8h}\label{sec:int}
\includeimage
{diagram_A.png} % Имя файла без расширения (файл должен быть расположен в директории inc/img/)
{f} % Обтекание (без обтекания)
{h} % Положение рисунка (см. figure из пакета float)
{0.65\textwidth} % Ширина рисунка
{Схема А} % Подпись рисунка
\clearpage

\includeimage
{diagram_B.png} % Имя файла без расширения (файл должен быть расположен в директории inc/img/)
{f} % Обтекание (без обтекания)
{h} % Положение рисунка (см. figure из пакета float)
{0.7\textwidth} % Ширина рисунка
{Схема Б} % Подпись рисунка
\clearpage

\includeimage
{diagram_C.png} % Имя файла без расширения (файл должен быть расположен в директории inc/img/)
{f} % Обтекание (без обтекания)
{h} % Положение рисунка (см. figure из пакета float)
{0.65\textwidth} % Ширина рисунка
{Схема С} % Подпись рисунка
\clearpage

\section{Схема прерывания sub\_6}\label{sec:sub}
\includeimage
{diagram_sub} % Имя файла без расширения (файл должен быть расположен в директории inc/img/)
{f} % Обтекание (без обтекания)
{h} % Положение рисунка (см. figure из пакета float)
{0.65\textwidth} % Ширина рисунка
{sub\_6} % Подпись рисунка

\end{document}
